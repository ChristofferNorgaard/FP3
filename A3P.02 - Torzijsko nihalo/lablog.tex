\documentclass[11pt, a4paper]{article}
\usepackage[utf8]{fontenc}
\usepackage[T1]{fontenc}
\usepackage{geometry}
\usepackage{amsmath, amssymb, bm}
\usepackage{physics, siunitx}
\usepackage{hyperref}
\usepackage[slovene]{babel}

% ipe support
\usepackage{tikz}
\usetikzlibrary{arrows.meta,patterns}
\usetikzlibrary{ipe} % ipe compatibility library


\geometry{margin=3.5cm}
\sisetup{separate-uncertainty=true, exponent-product=\cdot, range-units=single}
\hypersetup{colorlinks=true, linkcolor=blue, urlcolor=cyan}
\setlength{\parindent}{0pt}

\newcommand{\diff}{\mathop{}\!\mathrm{d}}
\newcommand{\TODO}[1]{{\textbf{TODO:} {\color{red} #1}}}

\begin{document}
\section{Torzijsko nihalo z visečo žico}
\subsection{Naloga}
Na jekleni žici imamo obešeno kovinsko ploščo z ročajem. Želimo izmeriti njen nihajni čas in s tem torzijske lasnosti žice.
\subsection{Potrebščine}
\begin{itemize}
    \item stojalo, jeklena žica, plošča z ročajem
    \item uteži: votel kovinski val, kvader z valjasto votlino
    \item tehtnica, štoparica, klunasto merilo, mikrometer
\end{itemize}

\subsection{Potek}    
\subsubsection{Geometrijske razsežnosti}
Izmeri lasnosti žice, ter obeh uteži. 

Lastnosti žice (vse tri širine izmeri na različnih mestih)
\begin{center}
\begin{tabular}{|l|l|l|l|l|}
\hline
Meritev & Dolžina {[}mm{]} & Širina 1 {[}mm{]} & Širina 2 {[}mm{]} & \begin{tabular}[c]{@{}l@{}}Širina 3\\ {[}mm{]}\end{tabular} \\ \hline
1       &                  &                   &                   &                                                             \\ \hline
2       &                  &                   &                   &                                                             \\ \hline
\end{tabular}
\end{center}

Prav tako skiciraj in stehtaj ostali dve telesi.
\pagebreak

\subsubsection{Izmeri nihajni čas teles}

Meri z štoparico. Meri 10 ponovitev. Ploščo izmakni za majhen kot.

\begin{center}
\begin{tabular}{|l|l|l|l|}
\hline
Meritev & Prazna ploščad [s] & Kvader [s] & Valj  [s] \\ \hline
1       &                &        &      \\ \hline
2       &                &        &      \\ \hline
3       &                &        &      \\ \hline
4       &                &        &      \\ \hline
5       &                &        &      \\ \hline
\end{tabular}
\end{center}
\end{document}

