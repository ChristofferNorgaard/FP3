\documentclass[11pt, a4paper]{article}
\usepackage[utf8]{fontenc}
\usepackage[T1]{fontenc}
\usepackage{geometry}
\usepackage{amsmath, amssymb, bm}
\usepackage{physics, siunitx}
\usepackage{hyperref}
\usepackage[slovene]{babel}

% ipe support
\usepackage{tikz}
\usetikzlibrary{arrows.meta,patterns}
\usetikzlibrary{ipe} % ipe compatibility library


\geometry{margin=3.5cm}
\sisetup{separate-uncertainty=true, exponent-product=\cdot, range-units=single}
\hypersetup{colorlinks=true, linkcolor=blue, urlcolor=cyan}
\setlength{\parindent}{0pt}

\newcommand{\diff}{\mathop{}\!\mathrm{d}}
\newcommand{\TODO}[1]{{\textbf{TODO:} {\color{red} #1}}}

\begin{document}
\section{Upogib}
\subsection{Namen vaje}
Namen vaje je izmeriti elastično deformacijo snovi v vzdolžni smeri. 
\subsection{Potrebščine}
\begin{itemize}

    \item stojalo, mikrometerska ura
    \item uteži, tehtnica, kljuka za obešanje uteži
    \item dve ravni palici okroglega in pravokotnega profila
    \item kljunasto merilo in meter

\end{itemize}
\subsection{Naloge}
\begin{itemize}

    \item Opazuj upogibanje dveh palic različnih presekov v odvisnosti od obremenitve in
izračunaj njuna prožnostna modul
\item Oceni maksimalno obremenitev palic ter za koliko se palici upogneta zaradi lastne
teže. Primerjaj tudi gostoti obeh palic
\item Nariši diagrama spreminjanja strižne sile in navora vzdolž palice za izbrano utež.

\end{itemize}
\subsection{Potek}
\subsubsection{Izmeri vse dimenzije palic in jih stehtaj}
Kvadratna palica:
\begin{center}
\begin{tabular}{|l|l|l|l|l|}
\hline
Meritev & Dolžina {[}cm{]} & Širina {[}cm{]} & Višina {[}cm{]} & Teža {[}g{]} \\ \hline
1       &                  &                 &                 &              \\ \hline
2       &                  &                 &                 &              \\ \hline
\end{tabular}
\end{center}

Okrogla palica
\begin{center}
\begin{tabular}{|l|l|l|l|}
\hline
Meritev & Dolžina {[}cm{]} & Širina {[}cm{]} & Teža {[}g{]} \\ \hline
1       &                  &                 &              \\ \hline
2       &                  &                 &              \\ \hline
\end{tabular}
\end{center}

Izmeri še razdaljo med nosilcema:
\begin{center}
\begin{tabular}{|l|l|}
\hline
Meritev & Dolžina {[}cm{]} \\ \hline
1       &                  \\ \hline
2       &                  \\ \hline
\end{tabular}
\end{center}

\subsubsection{Umeri silo mikrometra}
Mikrometer z svojo vzmetjo pritiska na palico. S pomočjo tehtnice, ki jo nastaviš tako, da mikrometer pritiska na tehtnico, izmeri silo v 10 različnih položajih.

\begin{center}
    \begin{tabular}{|l|l|l|}
\hline
Meritev & Dolžina [$ \mu m $] & Teža na tehtnici {[}g{]} \\ \hline
1       &                  &                          \\ \hline
2       &                  &                          \\ \hline
3       &                  &                          \\ \hline
4       &                  &                          \\ \hline
5       &                  &                          \\ \hline
6       &                  &                          \\ \hline
7       &                  &                          \\ \hline
8       &                  &                          \\ \hline
9       &                  &                          \\ \hline
10      &                  &                          \\ \hline
\end{tabular}
\end{center}

\subsubsection{Izmeri upogib krožne palica}

Na palico dodajaj uteži. Z mikrometrskim merilo odčitavaj položaje. Meri pri nalaganju in odlaganju uteži. 
\begin{center}
    \begin{tabular}{|l|l|ll|}
\hline
Meritev & Obešena teža {[}g{]} & \multicolumn{2}{l|}{Dolžina $ \mu m $}               \\ \hline
        &                      & \multicolumn{1}{l|}{Nalaganje} & Odlaganje \\ \hline
1       &                      & \multicolumn{1}{l|}{}          &           \\ \hline
2       &                      & \multicolumn{1}{l|}{}          &           \\ \hline
3       &                      & \multicolumn{1}{l|}{}          &           \\ \hline
4       &                      & \multicolumn{1}{l|}{}          &           \\ \hline
5       &                      & \multicolumn{1}{l|}{}          &           \\ \hline
6       &                      & \multicolumn{1}{l|}{}          &           \\ \hline
7       &                      & \multicolumn{1}{l|}{}          &           \\ \hline
8       &                      & \multicolumn{1}{l|}{}          &           \\ \hline
9       &                      & \multicolumn{1}{l|}{}          &           \\ \hline
10      &                      & \multicolumn{1}{l|}{}          &           \\ \hline
\end{tabular}
\end{center}
\subsubsection{Izmeri upogib kvadraste palica}

Na palico dodajaj uteži. Z mikrometrskim merilo odčitavaj položaje. Meri pri nalaganju in odlaganju uteži. 
\begin{center}
    \begin{tabular}{|l|l|ll|}
\hline
Meritev & Obešena teža {[}g{]} & \multicolumn{2}{l|}{Dolžina $ \mu m $}               \\ \hline
        &                      & \multicolumn{1}{l|}{Nalaganje} & Odlaganje \\ \hline
1       &                      & \multicolumn{1}{l|}{}          &           \\ \hline
2       &                      & \multicolumn{1}{l|}{}          &           \\ \hline
3       &                      & \multicolumn{1}{l|}{}          &           \\ \hline
4       &                      & \multicolumn{1}{l|}{}          &           \\ \hline
5       &                      & \multicolumn{1}{l|}{}          &           \\ \hline
6       &                      & \multicolumn{1}{l|}{}          &           \\ \hline
7       &                      & \multicolumn{1}{l|}{}          &           \\ \hline
8       &                      & \multicolumn{1}{l|}{}          &           \\ \hline
9       &                      & \multicolumn{1}{l|}{}          &           \\ \hline
10      &                      & \multicolumn{1}{l|}{}          &           \\ \hline
\end{tabular}
\end{center}

\end{document}

