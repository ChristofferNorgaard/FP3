\documentclass{article}
\usepackage{float}
\usepackage{graphicx} % Required for inserting images
\usepackage{babel}[slovene]
\usepackage{siunitx}
\usepackage{pgfplots}
\usepackage{pgfplotstable}
\usepackage{geometry}
\usepackage{amsmath}

\pgfplotsset{compat=1.18} 
\geometry{margin=1in}
\title{Poročilo vaje 20}
\author{Jakob Kralj}
\date{Oktober 2023}

\begin{document}

\begin{center}
\begin{tabular}{l@{\hskip 3cm}r@{\hskip 2cm}r}
\textbf{Fizikalni praktikum: Torzijsko Nihalo} & \textbf{Avtor:} & \textbf{Datum:} \\
& Jakob Kralj & \today \\
\end{tabular}
\end{center}

\section{Teoretični uvod}

Strižna napetost je napetost telesa preko ravnine, pravokotne na tlačno silo. Označimo jo kot
\begin{equation*}
    \frac{F}{S} = G \alpha,
\end{equation*}
kjer je $ F/S $ strižna napetost, $ G $pa strižni modul. Strižno napetost lahko izrazimo tudi pri torzijski deformaciji in sicer kot
\begin{equation*}
    M = D \phi,
\end{equation*}
kjer je $ M $ navor in $ \phi $ kot zasuka enega dela žice glede na drugega. Da bi izpeljali $ D $ glede na $ G $ si predstavljamo, da je žica sestavljena iz mnogih tankih cevk, ki se prilagajajo ena na drugo. Vsak majhen košček te cevke se ob torzijski deformaciji premakne za $ \phi r $, dolžina cevke pa je $ l $, torej se košček deformira za $ \alpha = r \phi / l $. Glede na zgornjo enačbo za strižno deformacija lahko torej napišemo:
\begin{equation*}
    dM = r dF = r \alpha G dS.
\end{equation*}

V enačbo vstavimo vrednosti za $ \alpha $ in $ S $, potem pa enačbo integriramo
\begin{equation} \label{eq:3}
    M = \int dM = \frac{\pi r_0^{2} G \phi}{2 l} =  D \phi,
\end{equation}
kar pomeni, da je $ D = \frac{\pi r_0^{2} G}{2l} $. $ G $ lahko povežemo z prožnostnim modulom preko:
\begin{equation}
G = \frac{E}{2(1 + \mu)},
\end{equation}
kjer je $ \mu $ Poissonovo število in predstavlja razmerje:
\begin{equation*}
    \frac{\Delta r}{r} = - \mu \frac{\Delta l}{l}.
\end{equation*}
Zadevo povežimo še z nihanjem na torzijskem nihalu. To opisuje naslednja enačba (za majhne odklone):
\begin{equation} \label{eq:2}
    t_0 = 2 \pi \sqrt{\frac{J}{D}},
\end{equation}

kjer je $ D $ že prej omenjeni torzijski koeficient, $ J $ pa vztrajnostni moment. V nalogi je potrebno izračunati vztrajnostni moment za dve telesi.

\paragraph{Votel valj}

Osnovna enačba za izračun vztrajnostnega momenta telesa je:
\begin{equation*}
   J = \rho \int_V r^{2} d V.
\end{equation*}

Glede na to, da gre za cilinder, izkoristimo simetrijo, da dobimo:
\begin{equation*}
    J = \rho \int_r^{R} r^{2} 2 \pi r L dr
\end{equation*}

Kar nam ob izvrednotenju da:
\begin{equation*}
    J = \frac{2 \pi L \rho (R^{4} - r^{4})}{4}
\end{equation*}

Enačbo lahko razpišemo preko razdelitve kvadratov v:
\begin{equation*}
    J = \frac{\pi L \rho (R^{2} - r^{2})(R^{2} + r^{2})}{2}
\end{equation*}

pri tem pa upoštevajmo, da je masa cilindra ravno $ M = \pi L \rho (R^{2} - r^{2}) $, kar nam da končno enačbo
\begin{equation} \label{eq:1}
    J = \frac{1}{2} M (R^{2} + r^{2})
\end{equation}

\paragraph{Kvader z valjasto votlino}

Pri izračunu lahko uporabimo dejstvo, da so vztrajnostni momenti aditivni. Tako je 
\begin{equation*}
    J_\text{votel kvader} = J_{\text{poln kvader}} - J_{\text{izrezan cilinder}}
\end{equation*}

Edina nevšečnost je torej določiti mase polnega kvadra in izrezanega cilindra, v kolikor imamo le maso votlega kvadra. 

Gostoto materiala, iz katerega je cilinder, lahko izračunamo kot:
\begin{equation*}
    \rho = \frac{M}{a \cdot b \cdot l - \pi r^{2} l},
\end{equation*}

vstavimo:
\begin{equation*}
J_\text{votel kvader} = \frac{1}{12} \rho (a \cdot b \cdot l) (a^{2} + b^{2}) - \frac{1}{2} \rho (\pi r^{2} l) r^{2}
\end{equation*}
in izpostavimo:
\begin{equation} \label{eq:5}
J_\text{votel kvader} = \frac{M}{12(a \cdot b \cdot l - \pi r^{2} l)} \left((a \cdot b \cdot l) (a^{2} + b^{2}) - 6(\pi r^{2} l) r^{2} \right).
\end{equation}

\section{Meritve}

\paragraph{Mase}
\begin{align*}
  &m_{\text{valja}}=2490.0 \pm 1 g && m_{\text{kvadra}}=1193.0 \pm 1 g &&m_{\text{zobnika}}=750 \pm 1 g
\end{align*}
\paragraph{Valj}
\begin{align*}
  &r_{\text{notranji}} = 7.2 \pm 0.05 mm && r_{\text{zunanji}} = 43.6 \pm 0.05 mm && h = 49.3 \pm 0.1 mm
\end{align*}
\paragraph{Kvader}
\begin{align*}
  &a = 60.0 \pm 0.1 mm && r_{\text{notranji}} = 20.1 \pm 0.05 mm
\end{align*}

\paragraph{Žica}
\begin{align*}
  &l = 20.8 \pm 0.1 cm && r = 0.27 \pm 0.005 mm \\
\end{align*}

\paragraph{Nihajni časi}

Tabela predstavlja meritev 10ih nihajnih časov:  \\
\begin{center}
\begin{tabular}{c|c|c|c|c}
Meritev & Prazna ploščad (s) & Kvader (s) & Valj (s) & Zobnik (s) \\
\hline
1 & 22,3 & 42,5 & 62,0 & 31,2 \\
2 & 21,7 & 42,3 & 61,9 & 31,1 \\
3 & 21,7 & 42,5 & 61,9 & 31,4 \\
4 & 21,5 & 42,2 & 61,7 & 31,3 \\
5 & 21,8 & 42,4 & 61,8 & 30,9
\end{tabular}
\end{center}

Iz izmerjenih časov lahko izračunamo:
\begin{align*}
  &t_{p} = 2.18 \pm 0.03 s && t_{k} = 4.238 \pm 0.013 s && t_{v} = 6.186 \pm 0.011 s &&  t_{z} = 3.118 \pm 0.019 s
\end{align*}

\section{Obdelava in rezultati}

Na podlagi enačbe \ref{eq:1}, in iz izmerjenih mas ter dimenzij, lahko izračunamo vztrajnostni moment valja kot \( J_{k} = 958000.0 \pm 11000.0 g mm² = 9.58 \cdot 10^{-4} \pm 1.1 \cdot 10^{-5}  kg m² \). Obrnjena enačba\ref{eq:2} nam pove, da
\begin{equation*}
  \frac{t_{0}^{2}  }{4 \pi^{2} } = \frac{J}{D},
\end{equation*}

Vemo pa, da je pri nihajnih časih votlega valja \( J_{s} = J_{p} + J_{v}  \):

\begin{equation*}
\frac{t_{v}^{2} - t_{p}^{2} }{4 \pi^{2} }  = \frac{J_{p}+J_{v}-J_{p} }{D} = \frac{J_{v} }{D},
\end{equation*}

enačbo obrnemo in izračunamo, da je \( D = 0.002864 \pm 1.8 \cdot 10^{-5} kg m^{2}  s^{-2}  \). Sedaj ko imamo \( D \), lahko iz istih enačb izračunamo še vztrajnostni moment plošče, ki je
\( J_{p} = 3.45 \cdot 10^{-4}  \pm 1.1 \cdot 10^{-5}  kg m² \).

Iz enačbe~\ref{eq:3} lahko izračunamo G, v kolikor poznamo \( D \), in sicer \( G = 7.14 \cdot 10^{10}  \pm 5.3 \cdot 10^{9}  kg m^{-1}  s^{-2}  \). Podatki iz literature za Jeklo se gibljejo od 72.5 Gpa do 85.3 GPa, kar pomeni, da se meriteve ne ujema znotraj napake.

Teoretično lahko izračunamo vztrajnostni moment votlega kvadra z enačbo~\ref{eq:5}. Dobimo \( J_{kT} = 9.744 \cdot 10^{-4}  \pm 2.8 \cdot 10^{-6}  kg m² \). Podatek dobimo tudi eksperimentalno, in sicer \( J_{k} = 9.58 \cdot 10^{-4} \pm 1.1 \cdot 10^{-6}  kg m²  \). Podatka se ne ujemata v okviru napake.

Podobno kot sem eksperimentalno dobil podatek za vztrajnostni moment kvadra, lahko izračunam še vztrajnostni moment zobnika. Dobim \( J_{z} = 3.61 \cdot 10^{-4}  \pm 1.2 \cdot 10^{-5}  kg m² \)
\end{document}
